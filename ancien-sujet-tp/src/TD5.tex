\chapter{TD5 : conversion de nombres en chaînes de caractères}

Cette série d'exercices devrait vous permettre de compléter par vous même les fonctions suivantes du fichier \ccode{DocumentUtil.c} :
\begin{itemize}
  \item \ccode{char * computeDocumentNumber(long id)} retourne une chaîne de caractère allouée sur le tas contenant la représentation en base 36 (0-9A-Z) du nombre paramètre;
  \item \ccode{char * formatDate(int day, int month, int year)} retourne une chaîne de caractère allouée sur le tas contenant la représentation de la date au format JJ/MM/AAAA (donc en base 10).
\end{itemize}

\begin{csourcecorrection}
char * computeDocumentNumber(long id) {
    char buf[1024];
    int pos = 0;
    int len = 0;
    long temp = id;
    long val;

    while (temp > 0) {
        len++;
        temp = temp / 36;
    }

    while (id > 0) {
        val = id % 36;
        if (val < 10)
            buf[len - 1 - pos] = (char) ('0' + val);
        else
            buf[len - 1 - pos] = (char) ('A' + val - 10);
        pos++;
        id = id / 36;
    }
    buf[pos] = '\0';
    return bridge_duplicateString(buf);
}
char * formatDate(int day, int month, int year) {
    char buf[1024];

    sprintf(buf, "%02d/%02d/%4d", day, month, year);
    return duplicateString(buf);
}
\end{csourcecorrection}


\section{Conversion base B vers décimale}

Soit $n$ un entier positif ou nul qui s'exprime en base $B$ ($B\geq2$) sous la forme de $N$ chiffres $b_0$, \ldots, $b_{N-1}$ ($b_i\in\{0,1,2,\ldots,B-1\}$).

\begin{center}
\begin{tabular}{|c|c|c|c|c|c|c|c|}
\hline
$b_{N-1}$&\ldots&$b_5$&$b_4$&$b_3$&$b_2$&$b_1$&$b_0$\\
\hline
\end{tabular}
\end{center}

On a :
\begin{itemize}
  \item $n=\sum_{i=0}^{N-1}b_iB^i$
  \item Schéma de Horner : $$n=\sum\limits_{i=0}^{N-1}b_iB^i=b_0+B[b_1+B[b_2+\ldots+B[b_{N-2}+Bb_{N-1}]]]]$$
  Exemple: 0101 => 1+2(0+2(1+2(0)))=4+1=5
\end{itemize}

\begin{warning}
\noindent Vous ne devez jamais calculer $B^i$ dans le schéma de Horner.
\end{warning}

\begin{enumerate}
  \item Écrire la fonction \ccode{BaseB2Dec} qui admet trois arguments (la base $B$, le nombre de chiffres $N$ du tableau, et un tableau de $N$ chiffres) et qui retourne l'entier long $n$ associé. On prendra soin de préciser la convention de représentation utilisée.

\begin{correction}
La convention de représentation retenue est la suivante :
\begin{center}
\begin{tabular}{|l|c|c|c|c|c|c|c|c|}
\hline
Indice du tableau C&N-1&\ldots&5&4&3&2&1&0\\
\hline
               &$b_{N-1}$&\ldots&$b_5$&$b_4$&$b_3$&$b_2$&$b_1$&$b_0$\\
\hline
\end{tabular}
\end{center}
Remarque : ce n'est pas la seule convention qui peut être retenue.
\end{correction}
\begin{csourcecorrection}
#include <math.h>
#include <stdio.h>

long int BaseB2Dec( int B, int N, int b[] )
{
  long int n;
  int i;

  n = b[N-1];
  for( i = N-2; i >= 0; --i)
    n = n*B + b[i];
  return n;
}
\end{csourcecorrection}
  \item Écrire la fonction \ccode{Bin2Dec}, utilisant \ccode{BaseB2Dec}, qui retourne la valeur décimale signée correspondant à la représentation binaire sur 16 bits fournie en paramètre.
  \begin{equation}
n=\left\{\begin{array}{ll}
      \sum\limits_{i=0}^{14}b_i*2^i&\mbox{ Si $n\geq 0$ (on a $b_{15}=0$)}\\
      -2^{15}+\sum\limits_{i=0}^{14}b_i*2^i&\mbox{ Si $n<0$ (on a $b_{15}=1$)}
      \end{array}\right.
  \label{dec2bin}
\end{equation}
\begin{csourcecorrection}
long int Bin2Dec( int b[] )
{
  long int n;

  n = Base2Dec(2,15,b);
  if (b[15] == 1)
    n = (-1L<<15) + n;
  return n;
}
\end{csourcecorrection}
  \item Écrire la fonction \ccode{main}
\begin{csourcecorrection}
int main(void)
{
  long int n;
  int b[16];
  int i;

  printf("Entrez les 16 bits du nombres :\n");
  for( i = 15; i >= 0; --i)
  {
    printf("bits %d : ",i);
    scanf("%d", &(b[i]));
  }
  n = Bin2Dec(b);
  printf("\nLe nombre vaut %ld", n);
  return 0;
}
\end{csourcecorrection}
\end{enumerate}


\section{Conversion décimale vers binaire}

Soit n un entier décimal de type \ccode{short}. Soit $b_0$, \ldots, $b_{15}$ sa forme binaire ($b_i\in\{0,1\}$).
\begin{center}
\begin{tabular}{|c|c|c|c|c|c|c|c|c|c|c|c|c|c|c|c|}
\hline
$b_{15}$&$b_{14}$&$b_{13}$&$b_{12}$&$b_{11}$&$b_{10}$&$b_9$&$b_8$&$b_7$&$b_6$&$b_5$&$b_4$&$b_3$&$b_2$&$b_1$&$b_0$\\
\hline
\end{tabular}
\end{center}

On a :
$$n=\begin{cases}
    \sum_{i=0}^{14}b_i2^i&\text{si }n\geq0\\
    -2^{15}+\sum_{i=0}^{14}b_i2^i&\text{si }n<0
    \end{cases}$$

Remarques :
\begin{itemize}
  \item n=1 => \textbf{0}000 0001
  \item n=-1 => 128-1=127 => \textbf{1}111 1111
  \item Soit $div$ l'opérateur de division entière et $mod$ l'opérateur reste de la division entière, alors pour tout entier $n$, on a : $n = (n\ div\ k)*k\ +\ (n\ mod\ k)$.
  \item Schéma de Horner sur un nombre non signé : $$n=\sum\limits_{i=0}^{N-1}b_iB^i=b_0+B[b_1+B[b_2+\ldots+B[b_{N-2}+Bb_{N-1}]]]]$$
    \begin{enumerate}
          \item $b_0=n\mod B$
          \item $b_1=(n\div B) \mod B$
          \item $b_2=((n\div B) \div B) \mod B$
          \item \ldots
      \end{enumerate}
    Exemple : $n=5$ $\Longrightarrow$ $0101$ en base $2$.
\end{itemize}

\begin{warning}
Vous ne devez jamais calculer $B^i$ dans le schéma de Horner.
\end{warning}
 

\begin{enumerate}
  \item Écrire la fonction \ccode{Dec2Bin} qui admet deux paramètres (un nombre entier de type \ccode{short} et un tableau de 16 entiers) et qui stocke dans ce tableau la représentation binaire signée de l'entier.


\begin{correction}
La même convention de représentation que l'exercice précédent est utilisée.

Il est important de remarquer que :
$$\sum_{i=0}^{14}b_i2^i=\begin{cases}
                          n&\text{si }b_{15}=0\text{ i.e. }n\geq0\\ 
                          n+-2^{15}&\text{si }b_{15}=1\text{ i.e. }n<0
                        \end{cases}$$
\end{correction}
\begin{csourcecorrection}
#include <stdio.h>
#include <stdlib.h>
#include <math.h> 

void Dec2Bin( short n, int b[] ) 
{
  int i;

  if (n<0)
  {
    b[15] = 1;
    n = (int) (pow(2,15) + n);
  }
  else
    b[15] = 0;

  for( i = 0; i < 15; i++)
  {
    b[i] = n % 2;
    n = (n -b[i]) / 2;
  }
}
\end{csourcecorrection}
\begin{correction}
Autre solution :
\end{correction}
\begin{csourcecorrection}
void Dec2Bin( short n, int b[] )
{
  int i;

  for( i = 0; i < 16; ++i)
    b[i] = (n & (1 << i)) > 0;
}
\end{csourcecorrection}
  \item Écrire la fonction \ccode{main} qui demande la saisie d'un nombre entier de type short et qui affiche sa représentation binaire.
\begin{csourcecorrection}
int main(void)
{
  int b[16];
  int n;
  int i;

  printf("Saisissez un nombre entier :");
  scanf("%hd", &n);
  Dec2Bin(n,b);
  printf("\n");
  for( i = 15; i >=0; --i)
    printf("%hd", b[i]);
  printf("\n");
  return 0;
}
\end{csourcecorrection}
\end{enumerate}



\section{Conversion d'une chaîne de caractères héxadécimale en un entier}

On considère la chaîne de caractères s qui représente un entier non signé sous forme hexadécimale. La chaîne s se présente sous une des formes suivantes : \ccode{"0X1AB97"}, \ccode{"0x1AB97"}, \ccode{"0x1aB97"} ou \ccode{"0X1Ab97"}.
Écrire la fonction admettant comme argument un pointeur sur la chaîne s et qui renvoie l'entier représenté par la chaîne s. Ecrire la fonction \ccode{main} associée.

\begin{correction}
Le principe est similaire à celui de la conversion de base B vers décimale.
\end{correction}
\begin{csourcecorrection}
#include <stdio.h>

int hexaToint( char * hexa )
{
  int n = 0;
  int v;
  int i;

  if (hexa[0] == '0')
    if ((hexa[1] == 'x') || (hexa[1] == 'X'))
    {
      i = 2;
      do
      {
        if (('0'<=hexa[i]) && (hexa[i]<='9'))
          v = hexa[i] - '0';
        else
          if (('a'<=hexa[i]) && (hexa[i]<='z'))
            v = hexa[i] - 'a'+10;
          else
            if (('A'<=hexa[i]) && (hexa[i] <= 'Z'))
              v = hexa[i] - 'A' + 10;
            else
              return n;
        n = n*16+v;
        ++i;
      }
      while (1);
    }
  return -1;
}

int main(void)
{
  char * h = "0x14dE1";
  char buf[200];
  int n = hexaToint(h);

  printf("%d\n", n); // affiche 85473

  gets(buf);
  n = hexaToint(buf);
  printf("%d\n", n);
  return 0;
}
\end{csourcecorrection}

\section{Conversion d'un entier en une chaîne de caractères}

Écrire une fonction qui admet deux paramètres (un entier n et un tableau s de caractères suffisamment grand), et qui écrit dans s la représentation de n sous forme d'une chaîne de caractères. Ecrire la fonction \ccode{main} associée. Lorsque le nombre n est négatif, le signe '-' doit précéder les nombres dans la chaîne s.

Les fonctions de \ccode{MyString.h} peuvent être utilisées.

\begin{correction}Le principe est similaire à celui de la conversion d'un entier en base B sauf que l'on stocke des caractères plutôt que des nombres.\end{correction}
\begin{csourcecorrection}
#include <stdio.h>
#include <stdlib.h>
#include <string.h>

void inverse( char * s );

char * intTostring( int n, char * s )
{
  char temp[2];
  int negatif;

  temp[1] = '\0';
  s[0] ='\0';
  if (n < 0)
  {
    negatif = 1;
    n = -n;
  }
  else
    negatif = 0;

  do
  {
    temp[0] = '0'+ (n % 10);
    n /= 10;
    strcat(s,temp);
  }
  while (n > 0);
  if (negatif)
    strcat(s,"-");

  inverse(s);
  return s;
}

void inverse( char * s )
{
  int i, j;
  int len = strlen(s);
  char c;

  for( i = 0, j = len-1; i<j; ++i, --j )
  {
    c = s[i];
    s[i] = s[j];
    s[j] = c;
  }
}

int main(void)
{
  char Result[20];

  intTostring(-102,Result);
  strcat(Result,"\n");
  printf(Result);
  return 0;
}
\end{csourcecorrection}

