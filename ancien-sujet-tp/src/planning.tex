\chapter{Organisation des séances}

Les séances de TDs et TPs vont vous permettre de progressivement écrire une
application de facturation ultra-simplifiée. Pour cela, les 14h TDs et 12h TPs
sont réparties de la façon suivante :

{\tabulinesep=1.2mm
\begin{tabu}[c]{|c|X|}
\hline
  TP 1.1, TP 1.2, TP 1.3 \& TP 1.4& Prise en main d'un IDE, d'un compilateur et du projet\\
\hline
  TD 2, TD 3 (1/2)& Allocation, caractères et chaines de caractères (strdup, strncpy, strcmp, strcasecmp, index, strstr, insertion, extraction)\\
\hline
  TD 3 (2/2)& Algorithme de Vigenère, calcul sur des caractères\\
\hline
  TP 4& Mise en oeuvre et finalisation des TD2 et TD3\\
\hline
  TD 5& Conversion de base et formattage de dates\\
\hline
  TD 6& Tableau dynamiques et E/S avec fichiers textes\\
\hline
  TD 7, TP 8& Fonctions de validation de valeur, conversions simples, E/S sur des fichiers binaires avec des enregistrements de taille fixe, attributs textes stockés dans l'enregistrement\\
\hline
  TP 9& Fonction de validation de valeur, conversions simples, E/S sur des fichiers binaires avec des enregistrements de taille fixe, attributs textes stockés sur le tas\\\hline 
  TD 10, TP 11& Liste chainée simple, opérations sur la liste, E/S sur des fichiers binaires avec des enregistrements de taille variable\\
\hline
  TD 12, TP 13& Lecture et analyse de fichiers textes avec longueur de lignes inconnues \textit{a priori}, gestion d'un dictionnaire de valeurs avec des unions, formattage de texte à l'aide du dictionnaire\\
\hline
\end{tabu}}


A la fin de ces séances vous devrez rendre le code que vous aurez produit afin
qu'il soit noté. Cette note constituera votre note de contrôle continu.
L'absentéisme fera également partie de la notation.

\begin{warning}
Pour que nous puissions étudier et corriger un maximum d'exercices lors des séances de travaux dirigés mais aussi pour les séances de travaux pratiques vous soient profitables, vous devez préparer les exercices avant les séances.
\end{warning}