\chapter{TD10\&TP11 : Manipulation d'un document}

Cette série d'exercices devrait vous permettre de compléter par vous même les fonctions de \ccode{DocumentUtil.c}, \ccode{DocumentRowList.c} et  \ccode{Document.c}.

\section{Fonctions génériques}

Pour les deux fonctions suivantes, on considère des fichiers binaires.

Ecrire la fonction \ccode{void writeString(const char * str, FILE * file)} qui écrit dans un fichier binaire la chaîne de caractères fournie en paramètre. La longueur de la chaîne est quelconque.

\begin{csourcecorrection}
void writeString(const char * str, FILE * file) {
    size_t len;
    len = stringLength(str);
    if (fwrite(&len, sizeof(len), 1, file) != 1)
        fatalError("writeString");
    if (len > 0)
        if (fwrite(str, len, 1, file) != 1)
            fatalError("writeString");
}
\end{csourcecorrection}


Ecrire la fonction \ccode{char * readString(FILE * file)} qui lit dans un fichier binaire une chaîne de caractères précédemment écrite via la fonction \ccode{writeString}. La chaîne de caractères retournée est allouée sur le tas.

\begin{csourcecorrection}
char * readString(FILE * file) {
    size_t len;
    char * result;

    if (fread(&len, sizeof(len), 1, file) != 1)
        fatalError("readString");
    result = (char*) malloc(len + 1);
    if (len > 0)
        if (fread(result, len, 1, file) != 1)
            fatalError("readString");
    result[len] = '\0';
    return result;
}
\end{csourcecorrection}

\section{La liste des produits du document}

La liste des produits d'un document est stockée sous la forme d'une liste chainée simple. Pour cela, on vous fournit la structure de données suivante :
\begin{csource}
/** Structure representing a row in a document (as a cell in a simple linked list) */
typedef struct _DocumentRow {
    char * code /** The code */;
    char * designation /** The designation */;
    double quantity /** The quantity */;
    char * unity /** The unity */;
    double basePrice /** The base price */;
    double sellingPrice /** The selling price */;
    double discount /** The discount */;
    double rateOfVAT /** The rate of VAT */;
    struct _DocumentRow * next /** The pointer to the next row */;
} DocumentRow;
\end{csource}

Lorsque plusieurs cellules sont chainées en mémoire, on obtient une organisation telle que :
\begin{center}
\pgfimage[width=0.8\textwidth]{listev2}
\end{center} 


\subsection{Manipulation des cellules}

\subsubsection{Initialisation d'une cellule de la liste}

Ecrire la fonction \ccode{void DocumentRow_init(DocumentRow * row)} qui initialise tous les champs d'une cellule à des valeurs \og raisonnable \fg{} de manière à représenter une ligne de produit vide.

\begin{csourcecorrection}
void DocumentRow_init(DocumentRow * row) {
    row->code = duplicateString("");
    row->designation = duplicateString("");
    row->quantity = 0;
    row->unity = duplicateString("");
    row->basePrice = 0;
    row->sellingPrice = 0;
    row->discount = 0;
    row->rateOfVAT = 0;
    row->next = NULL;
}
\end{csourcecorrection}


\subsubsection{Finalisation d'une cellule}

Ecrire la fonction \ccode{void DocumentRow_finalize(DocumentRow * row)} qui finalise une cellule en libérant toute la mémoire non nécessaire à la cellule.

\begin{csourcecorrection}
void DocumentRow_finalize(DocumentRow * row) {
    free(row->code);
    free(row->designation);
    free(row->unity);
}
\end{csourcecorrection}

\subsubsection{Allocation sur le tas et initialisation d'une cellule}

Ecrire la fonction \ccode{DocumentRow * DocumentRow_create( void )} qui alloue une cellule sur le tas, l'initialise et la retourne.

\begin{csourcecorrection}
DocumentRow * DocumentRow_create(void) {
    DocumentRow * row;

    row = (DocumentRow *) malloc(sizeof(DocumentRow));
    if (row == NULL)
        fatalError("DocumentRow_Create");
    DocumentRow_init(row);
    return row;
}
\end{csourcecorrection}

\subsubsection{Destruction d'une cellule allouée sur le tas}

Ecrire la fonction \ccode{void DocumentRow_destroy(DocumentRow * row)} qui desalloue une cellule précédement allouée par \ccode{DocumentRow_create}.

\begin{csourcecorrection}
void DocumentRow_destroy(DocumentRow * row) {
    DocumentRow_finalize(row);
    free(row);
}
\end{csourcecorrection}

\subsubsection{Ecriture d'une cellule dans un fichier binaire}

En utilisant la fonction \ccode{writeString}, écrire la fonction \ccode{void DocumentRow_writeRow(DocumentRow * row, FILE * file)} qui écrit le contenu d'une cellule dans un fichier binaire.

\begin{csourcecorrection}
void DocumentRow_writeRow(DocumentRow * row, FILE * file) {
    writeString(row->code, file);
    writeString(row->designation, file);
    if (fwrite(&row->quantity, sizeof(row->quantity), 1, file) != 1)
        fatalError("writeRow");
    writeString(row->unity, file);
    if (fwrite(&row->basePrice, sizeof(row->basePrice), 1, file) != 1)
        fatalError("writeRow");
    if (fwrite(&row->sellingPrice, sizeof(row->sellingPrice), 1, file) != 1)
        fatalError("writeRow");
    if (fwrite(&row->discount, sizeof(row->discount), 1, file) != 1)
        fatalError("writeRow");
    if (fwrite(&row->rateOfVAT, sizeof(row->rateOfVAT), 1, file) != 1)
        fatalError("writeRow");
}
\end{csourcecorrection}

\subsubsection{Lecture d'une cellule à partir d'un fichier binaire}

En utilisant la fonction \ccode{readString}, écrire la fonction \ccode{DocumentRow * DocumentRow_readRow(FILE * fichier)} qui retourne une cellule allouée sur le tas dont le contenu est lu à partir du fichier binaire.

\begin{csourcecorrection}
DocumentRow * DocumentRow_readRow(FILE * fichier) {
    DocumentRow * row;

    row = DocumentRow_create();
    free(row->code);
    row->code = readString(fichier);
    free(row->designation);
    row->designation = readString(fichier);
    if (fread(&row->quantity, sizeof(row->quantity), 1, fichier) != 1)
        fatalError("readRow");
    free(row->unity);
    row->unity = readString(fichier);
    if (fread(&row->basePrice, sizeof(row->basePrice), 1, fichier) != 1)
        fatalError("readRow");
    if (fread(&row->sellingPrice, sizeof(row->sellingPrice), 1, fichier) != 1)
        fatalError("readRow");
    if (fread(&row->discount, sizeof(row->discount), 1, fichier) != 1)
        fatalError("readRow");
    if (fread(&row->rateOfVAT, sizeof(row->rateOfVAT), 1, fichier) != 1)
        fatalError("readRow");
    return row;
}
\end{csourcecorrection}

\subsection{Manipulation de la liste}

\subsubsection{Initialisation d'une liste}

Une liste chainée est assimilable à une tête de liste et aux cellules composants la liste. Ecrire la fonction \ccode{void DocumentRowList_init(DocumentRow ** list)} qui initialise une liste comme étant vide.

\begin{csourcecorrection}
/** Initialize a list of rows
 * @param list the address of the pointer on the first cell of the list
 */
void DocumentRowList_init(DocumentRow ** list) {
    *list = NULL;
}
\end{csourcecorrection}

\subsubsection{Destruction de liste}

Ecrire la fonction \ccode{void DocumentRowList_finalize(DocumentRow ** list)} qui détuit une liste en détruisant les cellules de la liste. A la fin de la fonction, la liste doit être une liste valide mais vide.

\begin{csourcecorrection}
void DocumentRowList_finalize(DocumentRow ** list) {
    while (*list != NULL) {
        DocumentRowList_removeRow(list, *list);
    }
}
\end{csourcecorrection}


\subsubsection{Accès au n-ième élément de la liste}

Ecrire la fonction \ccode{DocumentRow * DocumentRowList_get(DocumentRow * list, int rowIndex)} qui re\-tour\-ne un pointeur sur le \ccode{rowIndex}-ième élément de la liste. Si cet élément n'existe pas, la liste renvoie \ccode{NULL}.

\begin{csourcecorrection}
DocumentRow * DocumentRowList_get(DocumentRow * list, int rowIndex) {
    DocumentRow * cur;
    int i;

    cur = list;
    i = 0;
    
    if (rowIndex < 0) 
        return NULL;

    while (cur != NULL && i != rowIndex) {
        cur = cur->next;
        i++;
    }
    return cur;
}
\end{csourcecorrection}

\subsubsection{Nombre d'éléments d'une liste}

Ecrire la fonction \ccode{int DocumentRowList_getRowCount(DocumentRow * list)} qui retoune le nombre d'éléments d'une liste.

\begin{csourcecorrection}
int DocumentRowList_getRowCount(DocumentRow * list) {
    int count = 0;

    while (list != NULL) {
        count++;
        list = list->next;
    }
    return count;
}
\end{csourcecorrection}

\subsubsection{Ajout en fin de liste}

Ecrire la fonction \ccode{void DocumentRowList_pushBack(DocumentRow ** list, DocumentRow * row)} qui a\-jou\-te une cellule à la fin de la liste.

\begin{csourcecorrection}
void DocumentRowList_pushBack(DocumentRow ** list, DocumentRow * row) {
    if (*list == NULL)
        *list = row;
    else {
        DocumentRow * cur = *list;
        while (cur->next != NULL)
            cur = cur->next;
        cur->next = row;
    }
}
\end{csourcecorrection}

\subsubsection{Insertion après une cellule}

Ecrire la fonction \ccode{DocumentRowList_insertAfter} qui insère une cellule après la position indiquée dans une liste.
\begin{csource}
void DocumentRowList_insertAfter(DocumentRow ** list, DocumentRow * position, DocumentRow * row);
\end{csource}

\begin{csourcecorrection}
void DocumentRowList_insertAfter(DocumentRow ** UNUSED(list), DocumentRow * position,
        DocumentRow * row) {
    row->next = position->next;
    position->next = row;
}
\end{csourcecorrection}

\subsubsection{Insertion avant une cellule}

Ecrire la fonction \ccode{DocumentRowList_insertBefore} qui insère une cellule avant la position indiquée dans une liste.
\begin{csource}
void DocumentRowList_insertBefore(DocumentRow ** list, DocumentRow* position, DocumentRow* row);
\end{csource}

\begin{csourcecorrection}
void DocumentRowList_insertBefore(DocumentRow ** list, DocumentRow * position, DocumentRow * row) {
    if (position != NULL) {
        if (position == *list) {
            row->next = position;
            *list = row;
        } else {
            DocumentRow * previous = *list;
            while (previous != NULL && previous->next != position)
                previous = previous->next;
            if (previous != NULL) {
                row->next = previous->next;
                previous->next = row;
            }
        }
    }
}
\end{csourcecorrection}


\subsubsection{Suppression d'une cellule}

Ecrire la fonction \ccode{void DocumentRowList_removeRow(DocumentRow ** list, DocumentRow * position)} qui supprime la cellule indiquée d'une liste.

\begin{csourcecorrection}
void DocumentRowList_removeRow(DocumentRow ** list, DocumentRow * position) {
    if (*list == position) {
        *list = position->next;
        DocumentRow_destroy(position);
    } else {
        DocumentRow * previous = *list;
        while (previous != NULL && previous->next != position)
            previous = previous->next;
        if (previous != NULL)
            previous->next = position->next;
        DocumentRow_destroy(position);
    }
}
\end{csourcecorrection}

\section{Le document}

Un document est défini grâce aux structures de données suivantes:
\begin{csource}
/** Enumeration defining the type of a document */
typedef enum {
    QUOTATION /**< It's a quotation */,
    BILL /**< It's a bill */
} TypeDocument;

/** Structure representing a document */
typedef struct {
    CustomerRecord customer /** The customer */;
    char * editDate /** The last edit data */;
    char * expiryDate /** The peremption date */;
    char * docNumber /** The document number */;
    char * object /** The object of the document */;
    char * operator /** The last operator */;
    DocumentRow * rows /** The rows */;
    TypeDocument typeDocument /** The type of document */;
} Document;
\end{csource}

\subsection{Initialisation}

Ecrire la fonction \ccode{void Document_init(Document * document)} qui initialise un document  de manière à ce qu'il soit vierge.

\begin{csourcecorrection}
void Document_init(Document * document) {
    CustomerRecord_init(&document->customer);
    document->editDate = duplicateString("");
    document->expiryDate = duplicateString("");
    document->docNumber = duplicateString("");
    document->object = duplicateString("");
    document->operator = duplicateString("");
    DocumentRowList_init(&document->rows);
    document->typeDocument = QUOTATION;
}
\end{csourcecorrection}

\subsection{Finalisation}

Ecrire la fonction \ccode{void Document_finalize(Document * document)} qui finalise un document en libérant les éventuelles zones mémoires allouées.

\begin{csourcecorrection}
void Document_finalize(Document * document) {
    CustomerRecord_finalize(&document->customer);
    free(document->editDate);
    free(document->expiryDate);
    free(document->docNumber);
    free(document->object);
    free(document->operator);
    DocumentRowList_finalize(&document->rows);
}
\end{csourcecorrection}


\subsection{Ecriture dans un fichier}

Ecrire la fonction \ccode{void Document_saveToFile(Document * document, const char * filename)} qui écrit le con\-te\-nu d'un document dans un fichier binaire. Vous devez au maximum utiliser les fonctions définies précédement.

\begin{csourcecorrection}
void Document_saveToFile(Document * document, const char * filename) {
    DocumentRow * cur;
    int count;
    FILE * fichier;

    count = DocumentRowList_getRowCount(document->rows);
    fichier = fopen(filename, "wb");

    CustomerRecord_write(&document->customer, fichier);
    writeString(document->docNumber, fichier);
    writeString(document->editDate, fichier);
    writeString(document->object, fichier);
    writeString(document->operator, fichier);
    writeString(document->expiryDate, fichier);

    if (fwrite(&document->typeDocument, sizeof(document->typeDocument), 1, fichier) != 1)
        fatalError("Document_SaveToFile");

    if (fwrite(&count, sizeof(count), 1, fichier) != 1)
        fatalError("Document_SaveToFile");

    cur = document->rows;
    while (cur != NULL) {
        DocumentRow_writeRow(cur, fichier);
        cur = cur->next;
    }

    fclose(fichier);
}
\end{csourcecorrection}

\subsection{Lecture à partir d'un fichier binaire}

Ecrire la fonction \ccode{void Document_loadFromFile(Document * document, const char * filename)} qui lit le con\-te\-nu d'un document à partir d'un fichier. Le document, fournis en paramètre, à remplir a été précédemment initialisé par la fonction \ccode{Document_init}.

\begin{csourcecorrection}
void Document_loadFromFile(Document * document, const char * filename) {
    int count, i;
    FILE * fichier;

    fichier = fopen(filename, "rb");
    if (fopen == NULL)
        fatalError("Echec ouverture fichier");

    CustomerRecord_read(&document->customer, fichier);
    free(document->docNumber);
    document->docNumber = readString(fichier);
    free(document->editDate);
    document->editDate = readString(fichier);
    free(document->object);
    document->object = readString(fichier);
    free(document->operator);
    document->operator = readString(fichier);
    free(document->expiryDate);
    document->expiryDate = readString(fichier);

    if (fread(&document->typeDocument, sizeof(document->typeDocument), 1, fichier) != 1)
        fatalError("Document_LoadFromFile");

    if (fread(&count, sizeof(count), 1, fichier) != 1)
        fatalError("Document_LoadFromFile");

    DocumentRowList_finalize(&document->rows);
    for (i = 0; i < count; ++i)
        DocumentRowList_pushBack(&document->rows, DocumentRow_readRow(fichier));

    fclose(fichier);
}
\end{csourcecorrection}

